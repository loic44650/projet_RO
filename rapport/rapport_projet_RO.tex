\documentclass[a4paper,sffamily,12pt]{article}

\usepackage[T1]{fontenc}
\usepackage[french]{babel}
\usepackage[utf8]{inputenc}

% Customization des listes
\usepackage{enumitem}
\usepackage{pifont}

% Insertion d'image
\usepackage{graphicx}
\usepackage{multirow}
\usepackage{xcolor}

% Création de lien
\usepackage[colorlinks,linkcolor=blue]{hyperref}

% Formatage des titres de sections
\usepackage{titlesec}
\titleformat{\section}
  {\normalfont\Large\bfseries\sffamily}{\thesection.}{0.33em}{}[\hrule]

 % En-tête
\usepackage{fancyhdr}
\pagestyle{fancy}
\renewcommand\headrulewidth{1pt}
\fancyhead[L]{Réseaux et télécoms}
\fancyhead[R]{$X6I0040$}

% Permet de mettre du texte au dessus du titre
\usepackage{titling}
\renewcommand{\maketitlehooka}{\noindent MAHIER Loïc \hfill groupe 601B\\ JEHANNO Clément \hfill \\}

% Titre
\title{\vspace{\fill}\LARGE\bfseries\sffamily Rapport de projet\protect\footnote{rapport réalisé sous \LaTeX} : Recherche opérationnelle\vspace{\fill}}

\begin{document}

	\date{} % Supprime la date
	\maketitle % Affiche le titre

	\thispagestyle{fancy} % Permet de mettre le titre sur la page ''fancy''
	
	\newpage
			
	\renewcommand{\contentsname}{Sommaire}
	\tableofcontents
	
	\newpage
	
	\section{Introduction}
	
		\vspace{0.5cm}

	\section{Partie algorithmique}
	
		\vspace{0.5cm}
		
		\subsection{Algorithme d'énumération des regroupements possible}
			
			\vspace{0.5cm}
		
		
	
		
		\subsection{Algorithme d'énumération des tournées}	
			
			\vspace{0.5cm}
	
	
	
			
	\section{Structures de données}		
					
			\vspace{0.5cm}
		
		
		
			
	\section{Analyse des résultats}
	
		\vspace{0.5cm}		
			
			
	\section{Améliorations}
	
		\vspace{0.5cm}		
			
			
			
	\section{Conclusion}
	
		\vspace{0.5cm}
										
\end{document}
